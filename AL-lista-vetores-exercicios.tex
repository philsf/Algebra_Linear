%\documentclass[pdftex,a4paper]{article}
\documentclass[a4paper]{article}
%%classes: article, report, book, proc, amsproc

%%%%%%%%%%%%%%%%%%%%%%%%
%% Misc
% para acertar os acentos
\usepackage[brazilian]{babel} 
%\usepackage[portuguese]{babel} 
% \usepackage[english]{babel}
% \usepackage[T1]{fontenc}
% \usepackage[latin1]{inputenc}
\usepackage[utf8]{inputenc}
\usepackage{indentfirst}
\usepackage{fullpage}
% \usepackage{graphicx} %See PDF section
\usepackage{multicol}
\setlength{\columnseprule}{0.5pt}
\setlength{\columnsep}{20pt}
%%%%%%%%%%%%%%%%%%%%%%%%
%%%%%%%%%%%%%%%%%%%%%%%%
%% PDF support

\usepackage[pdftex]{color,graphicx}
% %% Hyper-refs
\usepackage[pdftex]{hyperref} % for printing
% \usepackage[pdftex,bookmarks,colorlinks]{hyperref} % for screen

%% \newif\ifPDF
%% \ifx\pdfoutput\undefined\PDFfalse
%% \else\ifnum\pdfoutput > 0\PDFtrue
%%      \else\PDFfalse
%%      \fi
%% \fi

%% \ifPDF
%%   \usepackage[T1]{fontenc}
%%   \usepackage{aeguill}
%%   \usepackage[pdftex]{graphicx,color}
%%   \usepackage[pdftex]{hyperref}
%% \else
%%   \usepackage[T1]{fontenc}
%%   \usepackage[dvips]{graphicx}
%%   \usepackage[dvips]{hyperref}
%% \fi

%%%%%%%%%%%%%%%%%%%%%%%%


%%%%%%%%%%%%%%%%%%%%%%%%
%% Math
\usepackage{amsmath,amsfonts,amssymb}
% para usar R de Real do jeito que o povo gosta
\usepackage{amsfonts} % \mathbb
% para usar as letras frescas como L de Espaco das Transf Lineares
% \usepackage{mathrsfs} % \mathscr

% Oferecer seno e tangente em pt, com os comandos usuais.
\providecommand{\sin}{} \renewcommand{\sin}{\hspace{2pt}\mathrm{sen}}
\providecommand{\tan}{} \renewcommand{\tan}{\hspace{2pt}\mathrm{tg}}

% dt of integrals = \ud t
\newcommand{\ud}{\mathrm{\ d}}
%%%%%%%%%%%%%%%%%%%%%%%%



\begin{document}

%%%%%%%%%%%%%%%%%%%%%%%%
%% Título e cabeçalho
%\noindent\parbox[c]{.15\textwidth}{\includegraphics[width=.15\textwidth]{logo}}\hfill
\parbox[c]{.825\textwidth}{\raggedright%
  \sffamily {\LARGE

Álgebra Linear: Lista de Vetores

\par\bigskip}
{Prof: Felipe Figueiredo\par}
{\url{http://sites.google.com/site/proffelipefigueiredo}}

\vspace{1cm}
}
%%%%%%%%%%%%%%%%%%%%%%%%


%%%%%%%%%%%%%%%%%%%%%%%%

\begin{enumerate}
\item Calcule a norma dos seguintes vetores
  \begin{enumerate}
  \item $(0,1)$
  \item $(2,0)$
  \item $(1,2)$
  \item $(3,-2)$
  \item $(\sqrt{2},\sqrt{2})$
  \item $(\frac{\sqrt{3}}{2},\frac{1}{2})$
  \item $(0,0)$
  \item $(1,0,-1)$
  \item $(2,-\frac{1}{2},1)$
  \end{enumerate}
  
\item Calcule o produto escalar dos seguintes pares de vetores

  \begin{enumerate}
  \item $(1,1)$ e $(-1,2)$
  \item $(3,2)$ e $(2,-3)$
  \item $(-1,-2)$ e $(4,-6)$
  \item $(4,-6)$ e $(-1,-2)$
  \item $(\sqrt{2},\sqrt{2})$ e $(\sqrt{2},\sqrt{2})$
  \item $(\sqrt{2},2)$ e $(2\sqrt{3},\frac{1}{2})$
  \item $(-15,\sqrt{2})$ e $(0,0)$
  \item $(1,-1,2)$ e $(2,2,-1)$
  \item $(\sqrt{2},1,0)$ e $(-1,\sqrt{3},\sqrt{17})$

  \end{enumerate}

\item Esboce os vetores abaixo e calcule as projeções ortogonais pedidas
  \begin{enumerate}
  \item $u=(1,1)$ calcule $P^u_{OX}$
  \item $u=(2,3)$ calcule $P^u_{OY}$
  \item $u=(-1,3)$, $v=(1,1)$ calcule $P^u_{v}$
  \item $u=(-2,1)$, $v=(-1,-2)$ calcule $P^v_{u}$
  \item $u=(2,5)$, $v=(6,15)$ calcule $P^u_{v}$
  \item $u=(\sqrt{2},-\sqrt{3})$, $v=(2,-3)$ calcule $P^v_{u}$
  \item $u=(1,2,-1)$, $v=(0,-1,2)$ calcule $P^{u}_{v}$
  \item $u=(1,2,-1)$, $v=(0,-1,2)$ calcule $P^{v}_{u}$

  \end{enumerate}

\end{enumerate}
\end{document}
