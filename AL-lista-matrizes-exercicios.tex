%\documentclass[pdftex,a4paper]{article}
\documentclass[a4paper]{article}
%%classes: article, report, book, proc, amsproc

%%%%%%%%%%%%%%%%%%%%%%%%
%% Misc
% para acertar os acentos
\usepackage[brazilian]{babel} 
%\usepackage[portuguese]{babel} 
% \usepackage[english]{babel}
% \usepackage[T1]{fontenc}
% \usepackage[latin1]{inputenc}
\usepackage[utf8]{inputenc}
\usepackage{indentfirst}
\usepackage{fullpage}
% \usepackage{graphicx} %See PDF section
\usepackage{multicol}
\setlength{\columnseprule}{0.5pt}
\setlength{\columnsep}{20pt}
%%%%%%%%%%%%%%%%%%%%%%%%
%%%%%%%%%%%%%%%%%%%%%%%%
%% PDF support

\usepackage[pdftex]{color,graphicx}
% %% Hyper-refs
\usepackage[pdftex]{hyperref} % for printing
% \usepackage[pdftex,bookmarks,colorlinks]{hyperref} % for screen

%% \newif\ifPDF
%% \ifx\pdfoutput\undefined\PDFfalse
%% \else\ifnum\pdfoutput > 0\PDFtrue
%%      \else\PDFfalse
%%      \fi
%% \fi

%% \ifPDF
%%   \usepackage[T1]{fontenc}
%%   \usepackage{aeguill}
%%   \usepackage[pdftex]{graphicx,color}
%%   \usepackage[pdftex]{hyperref}
%% \else
%%   \usepackage[T1]{fontenc}
%%   \usepackage[dvips]{graphicx}
%%   \usepackage[dvips]{hyperref}
%% \fi

%%%%%%%%%%%%%%%%%%%%%%%%


%%%%%%%%%%%%%%%%%%%%%%%%
%% Math
\usepackage{amsmath,amsfonts,amssymb}
% para usar R de Real do jeito que o povo gosta
\usepackage{amsfonts} % \mathbb
% para usar as letras frescas como L de Espaco das Transf Lineares
% \usepackage{mathrsfs} % \mathscr

% Oferecer seno e tangente em pt, com os comandos usuais.
\providecommand{\sin}{} \renewcommand{\sin}{\hspace{2pt}\mathrm{sen}}
\providecommand{\tan}{} \renewcommand{\tan}{\hspace{2pt}\mathrm{tg}}

% dt of integrals = \ud t
\newcommand{\ud}{\mathrm{\ d}}
%%%%%%%%%%%%%%%%%%%%%%%%



\begin{document}

%%%%%%%%%%%%%%%%%%%%%%%%
%% Título e cabeçalho
%\noindent\parbox[c]{.15\textwidth}{\includegraphics[width=.15\textwidth]{logo}}\hfill
\parbox[c]{.825\textwidth}{\raggedright%
  \sffamily {\LARGE

Álgebra Linear: Lista de Matrizes

\par\bigskip}
{Prof: Felipe Figueiredo\par}
{\url{http://sites.google.com/site/proffelipefigueiredo}\par}
}

Versão: \verb|20141124|

%%%%%%%%%%%%%%%%%%%%%%%%

Nas questões a seguir, calcule o que se pede para cada item.

%%%%%%%%%%%%%%%%%%%%%%%%
\begin{multicols}{2}

\begin{enumerate}
\item Soma e produto por escalar

\begin{enumerate}
\item
$  \begin{bmatrix}
    5 & 5\\
    0 & 0 \\
  \end{bmatrix} +
  3 \begin{bmatrix}
    0 & 0\\
    1 & 1\\
  \end{bmatrix}
$

\item 
$
\begin{bmatrix}
  1 & 1\\
  1 & 1\\
\end{bmatrix}
- \mathrm{I}_2
$

\item 
$
2 \begin{bmatrix}
  1 & 2 & 3\\
  4 & 5 & 6\\
\end{bmatrix}
+
4 \begin{bmatrix}
 5 & 0 & 0\\
 0 & -1 & 0\\
\end{bmatrix}
$

\item 
$
3 \mathrm{I}_3 - 2 \mathrm{I}_3
$

\item 
$
-\frac{1}{2}\mathrm{0} +4
\begin{bmatrix}
  -1 & 0\\
  0 & 3\\
\end{bmatrix}
$

\item 
$ 
\begin{bmatrix}
  2 & 4\\
  2 & 4\\
  4 & 2\\
\end{bmatrix}
-
\begin{bmatrix}
  0 & -1\\
  0 & -2\\
  0 & -3\\
\end{bmatrix}
$

\end{enumerate}

\item Produto de matrizes

  \begin{enumerate}
\item 
$
\begin{bmatrix}
  1 & 2 & 0\\
  2 & 1 & 0\\
\end{bmatrix}
\cdot
\begin{bmatrix}
  1\\
  2\\
  3\\
\end{bmatrix}
$

\item 
$
\begin{bmatrix}
  2 & 1 & 0\\
  1 & 2 & 0\\
\end{bmatrix}
\cdot
\begin{bmatrix}
  1 & 2\\
  2 & 3\\
  3 & 4\\
\end{bmatrix}
$

\item 
$
\begin{bmatrix}
0 & 1\\
2 & 0\\
\end{bmatrix}
\cdot
(3\mathrm{I}_2)
$

\item 
$
\begin{bmatrix}
1 & -2\\
\end{bmatrix}
\cdot
\begin{bmatrix}
  4\\
  4\\
\end{bmatrix}
$

\item 
$
\begin{bmatrix}
4\\
4\\
\end{bmatrix}
\cdot
\begin{bmatrix}
  3 & 1\\
\end{bmatrix}
$

\item 
$
\begin{bmatrix}
-1 & 0 & 2\\
-2 & 0 & 1\\
\end{bmatrix}
\cdot
\begin{bmatrix}
  5 & 5\\
  1 & 1\\
\end{bmatrix}
$
\end{enumerate}

\item Mix

  \begin{enumerate}
  \item 
$
\begin{bmatrix}
1 & 1\\
1 & 1\\
\end{bmatrix}
\cdot
\begin{bmatrix}
  2 & 2\\
  2 & 2\\
\end{bmatrix}
+5
\begin{bmatrix}
  1 & 1\\
  2 & 2\\
\end{bmatrix}
$

  \item 
$
\begin{bmatrix}
 0 & 4\\
 0 & -2\\
\end{bmatrix}
-
\begin{bmatrix}
  2 & -3\\
  4 & 4\\
\end{bmatrix}
\cdot
\begin{bmatrix}
  0 & 2\\
  2 & 0\\
\end{bmatrix}
$

  \item 
$4
\begin{bmatrix}
  5 & 0 & 1\\
  0 & 2 & 0\\
  3 & 3 & 1
\end{bmatrix}
+
\begin{bmatrix}
  1\\
  1\\
  1\\
\end{bmatrix}
\cdot
\begin{bmatrix}
  -1 & 2 & -1
\end{bmatrix}
$

  \item 
$-10
\begin{bmatrix}
  1 & 1 & 1\\
  1 & 0 & 1\\
  0 & 1 & 0\\
\end{bmatrix}
\cdot
\begin{bmatrix}
  0 & 0 & 0\\
  1 & 0 & 1\\
  0 & 1 & 0\\
\end{bmatrix}
$

\end{enumerate}

\item Determinantes

  \begin{enumerate}
  \item 
$
\mathrm{I}_2
$

  \item 
$
3 \mathrm{I}_2
$

\item 
$
-10 \mathrm{I}_3
$

\item 
$
\begin{vmatrix}
  2 & 1\\
  1 & 2\\
\end{vmatrix}
$

\item 
$
\begin{vmatrix}
  1 & 1\\
  2 & -2\\
\end{vmatrix}
$

\item 
$
\begin{vmatrix}
1 & 0 & 1\\
0 & 1 & 0\\
1 & 0 & -1\\
\end{vmatrix}
$

\item 
$
\begin{vmatrix}
  \frac{1}{2} & \frac{1}{2}\\
  1 & -1\\
\end{vmatrix}
$

\item 
$
\begin{vmatrix}
  a & b & c\\
  2a & 2b & 2c\\
  0 & 1 & 0\\
\end{vmatrix}
$

\item 
$
\begin{vmatrix}
  3 & 0 & 0\\
  3 & 5 & 0\\
  0 & 0 & -1\\
\end{vmatrix}
$

\item 
$
\begin{vmatrix}
  2 & \sqrt{2} & 2\\
  0 & 2 & \sqrt{3}\\
  0 & 0 & 1\\
\end{vmatrix}
$

\end{enumerate}

\item Mix de determinantes

  \begin{enumerate}
\item 
$
\mathrm{det} \left(
\begin{bmatrix}
  2 & 0\\
  1 & 1\\
\end{bmatrix}
+
\begin{bmatrix}
  1 & 1\\
  0 & 1\\
\end{bmatrix}
\right)
$

\item 
$
\mathrm{det} \left(
\begin{bmatrix}
  3 & 1 & 5\\
  0 & 0 & 0\\
  0 & 0 & 1\\
\end{bmatrix}
\begin{bmatrix}
  1 & 2 & 3\\
  4 & 5 & 6\\
  7 &8 & 9\\
\end{bmatrix}
\right)
$

\item 
$
\mathrm{det} \left(
  2
  \begin{bmatrix}
   4 & 4\\
   1 & 2\\
  \end{bmatrix}
-5 \mathrm{I}_2
\right)
$

\item 
$
\mathrm{det} \left(
  \begin{bmatrix}
    5 & 2 & 100\\
    0 & 10 & 200\\
    0 & 0 & -1\\
  \end{bmatrix}
  \begin{bmatrix}
    0 & 1 & 0\\
    1 & 0 & 0\\
    0 & 0 & 1\\
  \end{bmatrix}
\right)
$

  \end{enumerate}

\end{enumerate}
  
\end{multicols}


% \section{Exercício 1}
% Considere as matrizes
% $A=
% \begin{bmatrix}
%   1 & 1\\
%   2 & 3\\
% \end{bmatrix}
% $,
% $B=\begin{bmatrix}
%   a & b\\
%   c & d\\
% \end{bmatrix}$,
% $C =
% \begin{bmatrix}
%   1 & 2 & 3\\
%   4 & 5 & 6\\
%   7 & 8 & 9\\
% \end{bmatrix}$,
% $D=
% \begin{bmatrix}
%   -2 & 0 & 1\\
%   3 & 5 & -1\\
%   -3 & -2 & -1\\
% \end{bmatrix}
% $,

% $E=
% \begin{bmatrix}
%   3 & 2 & -1\\
%   -5 & 5 & -5\\
% \end{bmatrix}
% $

\end{document}
