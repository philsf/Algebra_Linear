%\documentclass[pdftex,a4paper]{article}
\documentclass[a4paper]{article}
%%classes: article, report, book, proc, amsproc

%%%%%%%%%%%%%%%%%%%%%%%%
%% Misc
% para acertar os acentos
\usepackage[brazilian]{babel} 
%\usepackage[portuguese]{babel} 
% \usepackage[english]{babel}
% \usepackage[T1]{fontenc}
% \usepackage[latin1]{inputenc}
\usepackage[utf8]{inputenc}
\usepackage{indentfirst}
\usepackage{fullpage}
% \usepackage{graphicx} %See PDF section
\usepackage{multicol}
\setlength{\columnseprule}{0.5pt}
\setlength{\columnsep}{20pt}
%%%%%%%%%%%%%%%%%%%%%%%%
%%%%%%%%%%%%%%%%%%%%%%%%
%% PDF support

\usepackage[pdftex]{color,graphicx}
% %% Hyper-refs
\usepackage[pdftex]{hyperref} % for printing
% \usepackage[pdftex,bookmarks,colorlinks]{hyperref} % for screen

%% \newif\ifPDF
%% \ifx\pdfoutput\undefined\PDFfalse
%% \else\ifnum\pdfoutput > 0\PDFtrue
%%      \else\PDFfalse
%%      \fi
%% \fi

%% \ifPDF
%%   \usepackage[T1]{fontenc}
%%   \usepackage{aeguill}
%%   \usepackage[pdftex]{graphicx,color}
%%   \usepackage[pdftex]{hyperref}
%% \else
%%   \usepackage[T1]{fontenc}
%%   \usepackage[dvips]{graphicx}
%%   \usepackage[dvips]{hyperref}
%% \fi

%%%%%%%%%%%%%%%%%%%%%%%%


%%%%%%%%%%%%%%%%%%%%%%%%
%% Math
\usepackage{amsmath,amsfonts,amssymb}
% para usar R de Real do jeito que o povo gosta
\usepackage{amsfonts} % \mathbb
% para usar as letras frescas como L de Espaco das Transf Lineares
% \usepackage{mathrsfs} % \mathscr

% Oferecer seno e tangente em pt, com os comandos usuais.
\providecommand{\sin}{} \renewcommand{\sin}{\hspace{2pt}\mathrm{sen}}
\providecommand{\tan}{} \renewcommand{\tan}{\hspace{2pt}\mathrm{tg}}

% dt of integrals = \ud t
\newcommand{\ud}{\mathrm{\ d}}
%%%%%%%%%%%%%%%%%%%%%%%%



\begin{document}

%%%%%%%%%%%%%%%%%%%%%%%%
%% Título e cabeçalho
%\noindent\parbox[c]{.15\textwidth}{\includegraphics[width=.15\textwidth]{logo}}\hfill
\parbox[c]{.825\textwidth}{\raggedright%
  \sffamily {\LARGE

Álgebra Linear: Gabarito de Vetores

\par\bigskip}
{Prof: Felipe Figueiredo\par}
{\url{http://sites.google.com/site/proffelipefigueiredo}}

\vspace{1cm}
}
%%%%%%%%%%%%%%%%%%%%%%%%


%%%%%%%%%%%%%%%%%%%%%%%%

\begin{enumerate}
\item 

  \begin{enumerate}
  \item $1$
  \item $2$
  \item $\sqrt{5}$
  \item $\sqrt{13}$
  \item $2$
  \item $1$
  \item $0$ (Obs: vetor nulo = comprimento $0$)
  \item $\sqrt{2}$
  \item $\frac{\sqrt{21}}{2}$
  \end{enumerate}

\item 

  \begin{enumerate}
  \item $1$
  \item $0$ (Obs: vetores ortogonais)
  \item $8$
  \item $8$ (Obs: o produto escalar é comutativo, i.e., $u \cdot v = v
    \cdot u$
  \item $4$ (Obs: norma ao quadrado)
  \item $2\sqrt{6}+1$
  \item $0$ (Obs: o produto pelo vetor nulo sempre dá $0$)
  \item $-2$
  \item $\sqrt{3} - \sqrt{2}$
  \end{enumerate}

\item 
  \begin{enumerate}
  \item $P^{u}_{OX}=(1,0)$
  \item $P^{u}_{OY}=(0,3)$
  \item $P^{u}_{v}=(1,1)$
  \item $P^{v}_{u}=(0,0)$ (Obs: vetores ortogonais)
  \item $P^{u}_{v}=(2,5)$ (Obs: vetores colineares)
  \item $P^{v}_{u}=(\frac{2\sqrt{2} + 3\sqrt{3}}{5}) \cdot v =
    (\frac{4\sqrt{2} + 6\sqrt{3}}{5}, \frac{-6\sqrt{2} - 9\sqrt{3}}{5} )$
  \item $P^{u}_{v}=(0,\frac{4}{5}, - \frac{8}{5})$
  \item $P^{v}_{u}=(-\frac{2}{3},-\frac{4}{3},\frac{2}{3})$
  \end{enumerate}

\end{enumerate}
\end{document}
