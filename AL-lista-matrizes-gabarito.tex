%\documentclass[pdftex,a4paper]{article}
\documentclass[a4paper]{article}
%%classes: article, report, book, proc, amsproc

%%%%%%%%%%%%%%%%%%%%%%%%
%% Misc
% para acertar os acentos
\usepackage[brazilian]{babel} 
%\usepackage[portuguese]{babel} 
% \usepackage[english]{babel}
% \usepackage[T1]{fontenc}
% \usepackage[latin1]{inputenc}
\usepackage[utf8]{inputenc}
\usepackage{indentfirst}
\usepackage{fullpage}
% \usepackage{graphicx} %See PDF section
\usepackage{multicol}
\setlength{\columnseprule}{0.5pt}
\setlength{\columnsep}{20pt}
%%%%%%%%%%%%%%%%%%%%%%%%
%%%%%%%%%%%%%%%%%%%%%%%%
%% PDF support

\usepackage[pdftex]{color,graphicx}
% %% Hyper-refs
\usepackage[pdftex]{hyperref} % for printing
% \usepackage[pdftex,bookmarks,colorlinks]{hyperref} % for screen

%% \newif\ifPDF
%% \ifx\pdfoutput\undefined\PDFfalse
%% \else\ifnum\pdfoutput > 0\PDFtrue
%%      \else\PDFfalse
%%      \fi
%% \fi

%% \ifPDF
%%   \usepackage[T1]{fontenc}
%%   \usepackage{aeguill}
%%   \usepackage[pdftex]{graphicx,color}
%%   \usepackage[pdftex]{hyperref}
%% \else
%%   \usepackage[T1]{fontenc}
%%   \usepackage[dvips]{graphicx}
%%   \usepackage[dvips]{hyperref}
%% \fi

%%%%%%%%%%%%%%%%%%%%%%%%


%%%%%%%%%%%%%%%%%%%%%%%%
%% Math
\usepackage{amsmath,amsfonts,amssymb}
% para usar R de Real do jeito que o povo gosta
\usepackage{amsfonts} % \mathbb
% para usar as letras frescas como L de Espaco das Transf Lineares
% \usepackage{mathrsfs} % \mathscr

% Oferecer seno e tangente em pt, com os comandos usuais.
\providecommand{\sin}{} \renewcommand{\sin}{\hspace{2pt}\mathrm{sen}}
\providecommand{\tan}{} \renewcommand{\tan}{\hspace{2pt}\mathrm{tg}}

% dt of integrals = \ud t
\newcommand{\ud}{\mathrm{\ d}}
%%%%%%%%%%%%%%%%%%%%%%%%



\begin{document}

%%%%%%%%%%%%%%%%%%%%%%%%
%% Título e cabeçalho
%\noindent\parbox[c]{.15\textwidth}{\includegraphics[width=.15\textwidth]{logo}}\hfill
\parbox[c]{.825\textwidth}{\raggedright%
  \sffamily {\LARGE

Álgebra Linear: Gabarito de Matrizes

\par\bigskip}
{Prof: Felipe Figueiredo\par}
{\url{http://sites.google.com/site/proffelipefigueiredo}}

\vspace{1cm}
}
%%%%%%%%%%%%%%%%%%%%%%%%


%%%%%%%%%%%%%%%%%%%%%%%%
\begin{multicols}{2}

  \begin{enumerate}
  \item Soma e produto por escalar

    \begin{enumerate}
    \item
      $  \begin{bmatrix}
        5 & 5\\
        3 & 3\\
      \end{bmatrix}
      $

    \item 
      $
      \begin{bmatrix}
        0 & 1\\
        1 & 0\\
      \end{bmatrix}
      $

    \item 
      $
      \begin{bmatrix}
        22 & 4 & 6\\
        8 & 6 & 12\\
      \end{bmatrix}
      $

    \item $\mathrm{I}_3$
      
    \item 
      $
      \begin{bmatrix}
        -4 & 0\\
        0 & 12\\
      \end{bmatrix}
      $

    \item 
      $ 
      \begin{bmatrix}
        2 & 5\\
        2 & 6\\
        4 & 5\\
      \end{bmatrix}
      $

    \end{enumerate}

  \item Produto de matrizes

    \begin{enumerate}
    \item 
      $
      \begin{bmatrix}
        5\\
        4\\
      \end{bmatrix}
      $

    \item 
      $
      \begin{bmatrix}
        4 & 7\\
        5 & 8\\
      \end{bmatrix}
      $

    \item 
      $
      \begin{bmatrix}
        0 & 3\\
        6 & 0\\
      \end{bmatrix}
      $

    \item 
      $
      \begin{bmatrix}
        -4
      \end{bmatrix}
      $
      
    \item 
      $
      \begin{bmatrix}
        12 & 12\\
        4 & 4\\
      \end{bmatrix}
      $

    \item Não é possível multiplicar pois a primeira tem 3 colunas, e
      a segunda apenas 2 linhas.

    \end{enumerate}

    
  \item Mix

    \begin{enumerate}
    \item 
      $
      \begin{bmatrix}
        9 & 9\\
        14 & 14\\
      \end{bmatrix}
      $

    \item 
      $
      \begin{bmatrix}
        6 & 0\\
        -8 & -10\\
      \end{bmatrix}
      $

    \item 
      $
      \begin{bmatrix}
        19 & 2 & 3\\
        -1 & 10 & -1\\
        11 & 14 & 3\\
      \end{bmatrix}
      $

%      \columnbreak

    \item 
      $
      \begin{bmatrix}
        -10 & -10 & -10\\
        0 & -10 & 0\\
        -10 & 0 & -10\\
      \end{bmatrix}
      $

    \end{enumerate}

  \item Determinantes

    \begin{enumerate}
    \item $1$
      
    \item $ 3^2 = 9$

    \item $ -10^3 = -1000$

    \item $ 3 $
      
    \item $ -4 $

    \item $ -2 $

    \item $ -1 $

    \item $ 0 $ (Obs: A segunda linha é o dobro da primeira.)

    \item $  3 \cdot 5\cdot (-1) = -15 $ (Obs: matriz triangular)

    \item $ 2\cdot 2 \cdot 1 = 4 $ (Obs: matriz triangular)

    \end{enumerate}

  \item Mix de determinantes

    \begin{enumerate}
    \item $ 5 $

    \item $ \mathrm{detA} \cdot \mathrm{detB} = 0 \cdot \mathrm{detB}
      = 0 $ (Obs: A tem uma linha composta por zeros)

    \item $ -19 $

    \item $ \mathrm{detA} \cdot \mathrm{detB} = (-50) \cdot (-1) = 50
      $ (Obs: B é $\mathrm{I}_3$ com a primeira linha trocada pela
      segunda)

    \end{enumerate}

  \end{enumerate}
  
\end{multicols}



\end{document}
