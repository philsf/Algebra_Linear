%\documentclass[pdftex,a4paper]{article}
\documentclass[a4paper]{article}
%%classes: article, report, book, proc, amsproc

%%%%%%%%%%%%%%%%%%%%%%%%
%% Misc
% para acertar os acentos
\usepackage[brazilian]{babel} 
%\usepackage[portuguese]{babel} 
% \usepackage[english]{babel}
% \usepackage[T1]{fontenc}
% \usepackage[latin1]{inputenc}
\usepackage[utf8]{inputenc}
\usepackage{indentfirst}
\usepackage{fullpage}
% \usepackage{graphicx} %See PDF section
\usepackage{multicol}
\setlength{\columnseprule}{0.5pt}
\setlength{\columnsep}{20pt}
%%%%%%%%%%%%%%%%%%%%%%%%
%%%%%%%%%%%%%%%%%%%%%%%%
%% PDF support

\usepackage[pdftex]{color,graphicx}
% %% Hyper-refs
\usepackage[pdftex]{hyperref} % for printing
% \usepackage[pdftex,bookmarks,colorlinks]{hyperref} % for screen

%% \newif\ifPDF
%% \ifx\pdfoutput\undefined\PDFfalse
%% \else\ifnum\pdfoutput > 0\PDFtrue
%%      \else\PDFfalse
%%      \fi
%% \fi

%% \ifPDF
%%   \usepackage[T1]{fontenc}
%%   \usepackage{aeguill}
%%   \usepackage[pdftex]{graphicx,color}
%%   \usepackage[pdftex]{hyperref}
%% \else
%%   \usepackage[T1]{fontenc}
%%   \usepackage[dvips]{graphicx}
%%   \usepackage[dvips]{hyperref}
%% \fi

%%%%%%%%%%%%%%%%%%%%%%%%


%%%%%%%%%%%%%%%%%%%%%%%%
%% Math
\usepackage{amsmath,amsfonts,amssymb}
% para usar R de Real do jeito que o povo gosta
\usepackage{amsfonts} % \mathbb
% para usar as letras frescas como L de Espaco das Transf Lineares
% \usepackage{mathrsfs} % \mathscr

% Oferecer seno e tangente em pt, com os comandos usuais.
\providecommand{\sin}{} \renewcommand{\sin}{\hspace{2pt}\mathrm{sen}}
\providecommand{\tan}{} \renewcommand{\tan}{\hspace{2pt}\mathrm{tg}}

% dt of integrals = \ud t
\newcommand{\ud}{\mathrm{\ d}}
%%%%%%%%%%%%%%%%%%%%%%%%



\begin{document}

%%%%%%%%%%%%%%%%%%%%%%%%
%% Título e cabeçalho
%\noindent\parbox[c]{.15\textwidth}{\includegraphics[width=.15\textwidth]{logo}}\hfill
\parbox[c]{.825\textwidth}{\raggedright%
  \sffamily {\LARGE

Álgebra Linear: Lista de Sistemas Lineares

\par\bigskip}
{Prof: Felipe Figueiredo\par}
{\url{http://sites.google.com/site/proffelipefigueiredo}\par}
}

Versão: \verb|20141124|

%%%%%%%%%%%%%%%%%%%%%%%%


%%%%%%%%%%%%%%%%%%%%%%%%

\begin{enumerate}
\item Resolva os seguintes sistemas de equações lineares

  \begin{enumerate}
  \item $\left\{
      \begin{array}{ll}  
        x + \frac{y}{2} = 0\\
        -x + \frac{y}{2} =1 
      \end{array}
    \right.$
    
  \item $\left\{
      \begin{array}{ll}  
        2x +y =1\\
        x + 2y = 2
      \end{array}
    \right.$

  \item $\left\{
      \begin{array}{ll}  
        5x  + 10y = 0\\
        7x + 11y = 0
      \end{array}
    \right.$

  \item $\left\{
      \begin{array}{ll}  
        2x + 3y = 1\\
        6x +9y = 2
      \end{array}
    \right.$

  \item $\left\{
      \begin{array}{ll}  
        \frac{x}{2} + \frac{y}{3} = \frac{11}{2}\\
        -x + \frac{y}{4} = 0
      \end{array}
    \right.$

  \item $\left\{
      \begin{array}{ll}  
        -x + \frac{15}{13}y = 0\\
        x - \frac{15}{13}y = 0
      \end{array}
    \right.$

  \item $\left\{
      \begin{array}{ll}  
        x + y = 1\\
        x - y = 2\\
        x + 2y = 0
      \end{array}
    \right.$

  \item $\left\{
      \begin{array}{ll}  
        x + y + z = 1\\
        x + y - z = -1
      \end{array}
    \right.$

  \item $\left\{
      \begin{array}{ll}  
        x + y + z =1\\
        x -y -z = 1\\
        x +y -z = 2
      \end{array}
    \right.$

  \item $\left\{
      \begin{array}{ll}  
        x + 2y +z = -1\\
        x + y +2z = 0\\
        2x + y +z =1
      \end{array}
    \right.$

  \item $\left\{
      \begin{array}{ll}  
        x +\frac{y}{2} + z = 1\\
        -x + \frac{y}{2} -2z = 2\\
        -x + y + z = -2
      \end{array}
    \right.$

  \end{enumerate}
\end{enumerate}

Sugestão: Use o método de escalonamento em todos os exercícios. Na
primeira etapa, elimine $x$ na segunda equação usando o pivô da
primeira equação. Na segunda etapa, elimine $x$ na terceira equação
usando o pivô da primeira equação. Na terceira etapa, elimine $y$ da
terceira equação usando o pivô da segunda equação.

\end{document}
